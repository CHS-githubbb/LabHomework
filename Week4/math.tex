\documentclass[UTF8]{ctexart}
\usepackage{graphicx} % includegraphics
\usepackage{amsfonts} % mathbb
\usepackage{amsmath} % bmatrix
\usepackage{ctex}
\usepackage{geometry}
\usepackage{float}
\usepackage{enumerate}
%\geometry{left=2cm,right=2cm,top=2cm,bottom=5cm,scale=0.8}
\geometry{scale=0.8}

\title{数学分析与线性代数例题}
\author{佚名}
\date{2019 年 12 月 6 日}


\begin{document}

\maketitle
\tableofcontents

\section{微分中值定理及其应用}

\paragraph{定理1}(极值的第二充分条件). 设 $f(x)$在$(x_0-\delta,x_0+\delta)$ 可导且 
$f^{'}(x_0)=0$ ,又 $f^{''}(x_0)=0$ 存在.
\begin{enumerate}[1)]
\setlength{\itemindent}{1em}
    \item 若 $f^{''}(x_0)<0$ ,则 $f^{'}(x_0)$ 是严格极大值
    \item 若 $f^{''}(x_0)>0$ ,则 $f^{'}(x_0)$ 是严格极小值.
\end{enumerate}

\paragraph{例 1.}求 $y=\frac{1}{3}x\sqrt[3]{(x-5)^2}$ 的极值点与极值\footnote{原题摘自《数学分析简明教程》(上册)P142.}.
\paragraph{解.}函数在 $(-\infty,+\infty)$ 上连续,当 $x\not=5时$ 有
\begin{equation}
    y^{'}=\frac{1}{3}\left(\left(x-5\right)^\frac{2}{3}+\frac{2x}{3}\left(x-5\right)^\frac{-1}{3}\right)=\frac{5\left(x-3\right)}{9\left(x-5\right)^\frac{1}{3}}
\end{equation}
令 $y^{'}=0$ 得稳定点 $x=3$ ,现列表如下:
\begin{table}[H]
    \centering
    \begin{tabular}{|c|c|c|c|c|c|}
            \hline
        $x$ &$(-\infty,3)$ &$3$ &$(3,5)$ &$5$ &$(5,+\infty)$\\
            \hline
        $y^{'}$ &$+$ &$0$ &$-$ &$不存在$ &$+$\\
            \hline
        $y$ &$\nearrow$ &$\sqrt[3]{4}$ &$\searrow$ &$0$ &$\nearrow$\\
            \hline
    \end{tabular}  
    \label{tab1}
\end{table}

    从表中可见 $x=3$ 是极大值点,极大值为 $f(3)=\sqrt[3]{4}\mbox{;}x=5$ 为极小值点,极小值为 $f(5)=0$ . 我
们可以大致地画出函数的图形,如图\ref{fig:1}所示.
\newpage
\begin{figure}
    \centering\includegraphics[width=8cm]{function.pdf}
    \caption{$y=\frac{1}{3}x\sqrt[3]{(x-5)^2}$的函数图像}
    \label{fig:1}
\end{figure}



\section{行列式}
\paragraph{例 2. }若 $a,b\in\mathbb{R}^+$ ,求由方程为 $\frac{x_1^2}{a^2}+\frac{x_2^2}{b^2}=1$ 的椭圆为边界的区域 $E$ 的面积\footnote{原题摘自《线性代数及其应用》(第三版)P183.}.
\paragraph{解.}断言 $E$ 是单位圆盘 $D$ 在线性变换 $T$ 下的像. 这里 $T$ 由矩阵 $A=\begin{bmatrix} a&0\\ 0&b \end{bmatrix}$ 确定,这是因为若 
$\mathbf{u}=\begin{bmatrix} u_1\\u_2 \end{bmatrix}\mbox{,}\mathbf{x}=\begin{bmatrix} x_1\\x_2 \end{bmatrix}$ ,且 $\mathbf{x}=A\mathbf{u}$ ,则
\[u_1=\frac{x_1}{a},u_2=\frac{x_2}{b}\]
从而得 $u$ 在此单位圆内,即满足 $u_1^2+u_2^2\leq1$ ,当且仅当 $x$ 在 $E$ 内,即满足 $(x_1/a)^2+(x_2/b)^2\leq1$ . 进
而
 \begin{align*}
     \begin{split}
 \left\{ \mbox{椭圆的面积} \right\}  &= \left\{ \mbox{T(D)的面积} \right\}\\
                                    &= |\det A| \cdot\left\{ \mbox{D的面积} \right\}\\
                                    &= a \cdot b \cdot \pi \cdot(1)^2\\
                                    &= \pi a b
     \end{split}
 \end{align*}



\end{document}